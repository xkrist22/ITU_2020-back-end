\documentclass{beamer}
\usepackage[czech]{babel}
\usepackage[utf8]{inputenc}
\usetheme{CambridgeUS}
\usecolortheme{dove}
\title{Projekt Hra "Lodě"}
\subtitle{Tvorba uživatelských rozhraní}
\author{xkrist22, xhujic00, xpolis04}
\institute{FIT VUT}
\date{\today}
\begin{document}
	\begin{frame}
		\titlepage
	\end{frame}
	
	\begin{frame}
		\frametitle{Zadání}
Implementujte počítačovou variantu známé hry "Lodě" pro čtverečkovaný papír. Hra bude podporovat síťovou hru dvou hráčů. V první fázi hra umožní spojení dvou hráčů, ve druhé fázi bude probíhat rozmístění několika druhů lodí do hracího pole a třetí fází bude samotná hra (souboj). Aplikace bude zobrazovat dvě hrací pole: hráčovy lodě a protihráčovo bitevní pole, také by měla umožňovat komunikaci mezi hráči během hraní. 
	\end{frame}		

	\begin{frame}
		\frametitle{Rozdělení práce}
		\begin{itemize}
			\item back-end -- vytvoření logiky hry
			\item back-end -- "střelba" přes síť
			\item back-end -- hráčský chat
			\item front-end -- úvodní menu a propojení dvou hráčů
			\item front-end -- rozmístění lodí
			\item front-end -- souboj
		\end{itemize}
	\end{frame}		
			
	\begin{frame}
		\frametitle{Využívané platformy}
		\framesubtitle{Programovací jazyk}
		\begin{figure}[h]
		\centering
			\includegraphics[scale=0.5]{python_logo.png}
		\end{figure}
	\end{frame}

	\begin{frame}
		\frametitle{Využívané platformy}
		\framesubtitle{Vývoj GUI}
		\begin{itemize}
			\item Qt
			\item Tkinter
		\end{itemize}
	\end{frame}				
			
	\begin{frame}
		\frametitle{Pravidla v týmu}
		\begin{itemize}
			\item Členové alespoň jednou za den kontrolují komunikační kanál
			\item Komunikačním kanálem je Discord, případně školní e-mail
			\item Při výpadku bude práce rozdělena mezi zbývající členy
			\item Při rozhodování je potřeba společný souhlas
		\end{itemize}
	\end{frame}				
			
	\begin{frame}
		\frametitle{Prostor pro otázky}
	\end{frame}
\end{document}
